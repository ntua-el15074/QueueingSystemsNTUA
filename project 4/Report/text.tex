\documentclass[12pt]{article}
\usepackage[greek,english]{babel}
\usepackage{alphabeta}
\usepackage{listings}
\usepackage{xcolor}
\usepackage{hyperref}
\usepackage{tabularx}
\usepackage{mathtools}
\usepackage{graphicx}
\usepackage{blindtext}
\usepackage{geometry}
\usepackage{listings}
\usepackage{amsmath}
\usepackage{amsfonts}
\usepackage{steinmetz}
\usepackage{algorithm}
\usepackage[noend]{algpseudocode}
\usepackage[shortlabels]{enumitem}
\usepackage{tikz}
\usepackage{fdsymbol}
\geometry{
    a4paper,
    total={170mm,257mm},
    left=20mm,
    top=20mm,
}

\author{Αυγερινός Πέτρος 03115074}
\title{Συστήματα Αναμονής Άσκηση 4^η}
\date{}

\begin{document}
\maketitle 
\pagebreak

\tableofcontents
\pagebreak

\section{Ανάλυση και Σχεδιασμός τηλεφωνικού κέντρου}

\subsection{Ερώτημα 1}
Το διάγραμμα ρυθμού μεταβάσεων ενός συστήματος M/M/c/c είναι το παρακάτω:

\begin{center}
    \includegraphics[width=0.7\textwidth]{../images/mmcc.png}
\end{center}

Από τις εξισώσεις ισορροπίας γνωρίζουμε ότι:

\begin{equation}
    P_k = \frac{\lambda}{\mu k} P_{k-1} = 
    \frac{\lambda}{\mu k} \cdot \frac{\lambda}{\mu (k-1)} P_{k-2} =
    \ldots = \frac{(\frac{\lambda}{\mu})^k}{k!} P_0
\end{equation}

και 

\begin{equation}
    P_0 + P_1 + \ldots + P_c = 1
\end{equation}

Από (1) και (2) προκύπτει ότι:

\begin{align*}
        P_0 + P_1 + \ldots + P_c = 1 \\
        P_0 + \frac{\lambda}{\mu}P_0 + \ldots + \frac{(\frac{\lambda}{\mu})^c}{c!}P_0 = 1 \\
        P_0(1 + \frac{\lambda}{\mu} + \ldots + \frac{(\frac{\lambda}{\mu})^c}{c!}) = 1 \\
        P_0 \sum_{k=0}^{c} \frac{(\frac{\lambda}{\mu})^k}{k!} = 1 \\
        P_0 = \frac{1}{\sum_{k=0}^{c} \frac{\rho^k}{k!}}
\end{align*}

Η πιθανότητα απόρριψης είναι ίση με: 

\begin{align*}
    P_{blocking} = P_c = B(\rho,c) = \frac{(\frac{\lambda}{\mu})^c}{c!} P_0 = \\
    \frac{\rho^c}{c!} \frac{1}{\sum_{k=0}^{c} \frac{\rho^k}{k!}} \\
\end{align*}

Ο μέσος ρυθμός απωλειών από την ουρά είναι ίσος με:
\begin{equation}
    \lambda - \gamma = \lambda - \lambda (1 - P_{blocking}) = \lambda P_{blocking} = \lambda \frac {\rho^c}{c!} \frac{1}{\sum_{k=0}^{c} \frac{\rho^k}{k!}}
\end{equation}

Ο κώδικας για την συνάρτηση erlangb\_factorial φαίνεται παρακάτω:

\lstinputlisting[language=matlab, basicstyle=\tiny,
frame=single, backgroundcolor = \color{lightgray}]{../Code/erlangb_fact.m}

Το αποτέλεσμα του factorial για τις τιμές $9,9$ είναι ίδιες και ίσες με $0.2243$.

\subsection{Ερώτημα 2}
Για την απόδειξη της iterative formula αρχικά γνωρίζουμε ότι αν δεν υπάρχει 
εξυπηρετητής όλες οι κλήσεις απορρίπτονται. Επομένως:

\begin{equation}
    B(\rho,0) = 1
\end{equation}

Για $c - 1$ εξυπηρετητές έχουμε:

\begin{equation}
    B(\rho,c-1) = \frac{\rho^{c-1}}{(c-1)!} \frac{1}{\sum_{k=0}^{c-1} \frac{\rho^k}{k!}}
\end{equation}
\\

Προσθέτοντας έναν ακόμα εξυπηρετητή έχουμε:

\begin{align*}
    Β(\rho,c) = \frac{\rho^c}{c!} \frac{1}{\sum_{k=0}^{c} \frac{\rho^k}{k!}} = \\
    \frac{\frac{\rho^{c-1}\cdot \rho}{(c-1)! \cdot c}}{\frac{\rho^c}{c!} + \sum_{k=0}^{c-1} \frac{\rho^k}{k!}} = \\
    \frac{\frac{\rho^{c-1}}{(c-1)!} \cdot \frac{\rho}{c}}{\frac{\rho}{c} \cdot \frac{\rho^{c-1}}{(c-1)!} + \sum_{k=0}^{c-1} \frac{\rho^k}{k!}} = \\
    \frac{B(\rho,c-1) \cdot \frac{\rho}{c}}{B(\rho,c-1) \cdot \frac{\rho}{c} + 1} = \\
    \frac{\rho \cdot B(\rho,c-1)}{\rho \cdot B(\rho,c-1) + c}
\end{align*}
\\



Ο κώδικας για την συνάρτηση erlangb\_iterative φαίνεται παρακάτω:

\lstinputlisting[language=matlab, basicstyle=\tiny,
frame=single, backgroundcolor = \color{lightgray}]{../Code/erlangb_it.m}

Η iterative λύση δίνει το ίδιο αποτέλεσμα με το factorial για τις τιμές $9,9$ και είναι
ίσο με $0.2243$.

\subsection{Ερώτημα 3}
Στην περίπτωση του factorial παίρνουμε NaN ενώ στην περίπτωση του iterative παίρνουμε 
το αποτέλεσμα $0.0245243$. Αυτό συμβαίνει γιατί το factorial παίρνει πολύ μεγάλες τιμές
και υπερχειλίζει.

\subsection{Ερώτημα 4}
\begin{enumerate}
    \item{}
        Η συνολική κυκλοφοριακή ένταση του δικτύου για τον πιο απαιτητικό χρήστη είναι:
        \begin{equation}
            \rho = \frac{500 \cdot 23}{60} = 191.66 \text{ Erlangs}
        \end{equation}

    \item{}
        Τα διαγράμματα απόρριψης για τις τρεις τιμές λεπτών φαίνονται παρακάτω:

        Για 23 λεπτά:
        \begin{center}
            \includegraphics[width=0.5\textwidth]{../images/erlangb-23.png}
        \end{center}

        Με ζουμ:
        \begin{center}
            \includegraphics[width=0.5\textwidth]{../images/erlangb-23-zoom.png}
        \end{center}
        \\

        Για 34 λεπτά:
        \begin{center}
            \includegraphics[width=0.5\textwidth]{../images/erlangb-34.png}
        \end{center}

        Με ζουμ:
        \begin{center}
            \includegraphics[width=0.5\textwidth]{../images/erlangb-34-zoom.png}
        \end{center}
        \\


        Για 46 λεπτά:
        \begin{center}
            \includegraphics[width=0.5\textwidth]{../images/erlangb-46.png}
        \end{center}

        Με ζουμ:
        \begin{center}
            \includegraphics[width=0.5\textwidth]{../images/erlangb-46-zoom.png}
        \end{center}

    \item{}
        Προσθέτοντας μία συνθήκη για πιθανότητα μικρότερη του $0.01$ στον κώδικα βλέπουμε ότι:

        \begin{enumerate}
            \item{}
                Για 23 λεπτά, οι εξυπηρετητές που απαιτούνται είναι 8.

            \item{}
                Για 34 λεπτά, οι εξυπηρετητές που απαιτούνται είναι 17.

            \item{}
                Για 46 λεπτά, οι εξυπηρετητές που απαιτούνται είναι 54.
        \end{enumerate}
\end{enumerate}

\subsection{Ερώτημα 5}
\begin{enumerate}
    \item{}
        Η μέση τιμή της συνολικής προσφερόμενης κίνησης είναι $A = 20000 \cdot 0.06 = 1200$ Erlangs.
    \item{}
        Η κίνηση υπεραστικών είναι $Α' = 0.05Α = 60$ Erlangs. Για GoS $0.01$, με χρήση ενός 
        Erlang B calculator παίρνουμε ότι τα trunks που απαιτούνται είναι $75$.

    \item{}
        Με τριπλασιασμό του φορτίου η κίνηση υπεραστικών είναι $Α'' = 3Α' = 180$ Erlangs.
        Με χρήση του Erlang B calculator για $75$ trunks παίρνουμε GoS $0.587$.
\end{enumerate}

Ο κώδικας για τα παραπάνω ερωτήματα φαίνεται παρακάτω:

\lstinputlisting[language=matlab, basicstyle=\tiny,
frame=single, backgroundcolor = \color{lightgray}]{../Code/ask1.m}
\pagebreak

\section{Σύστημα εξυπηρέτησης με δύο ανόμοιους εξυπηρετητές}

\subsection{Ερώτημα 1}

Το διάγραμμα ρυθμών μεταβάσεων στην κατάσταση ισορροπίας για το δοσμένο σύστημα 
φαίνεται παρακάτω για $\lambda = 2, p = 1$ και $\mu_1 = 1.25, \mu_2 = 0.4$.

\begin{center}
    \includegraphics[width=0.7\textwidth]{../images/diagram_transfer_rate.png}
\end{center}

Οι εξισώσεις του συστήματος για ισορροπία είναι: 

\begin{equation}
    \lambda P_0 = \mu_1 P_{11} + \mu_2 P_{12} \Rightarrow 2P_0 = 1.25P_{11} + 0.4P_{12} 
\end{equation}

\begin{equation}
    (\lambda + \mu_1) P_{11}  = p \lambda P_0 + \mu_2 P_2 \Rightarrow 3.25 P_{11} = 2P_0 + 0.4P_2
\end{equation}

\begin{equation}
    (\lambda + \mu_2) P_{12} = (1-p) \lambda P_0 + \mu_1 P_2 \Rightarrow 2.4 P_{12} = 1.25P_2
\end{equation}

\begin{equation}
    P_0 + P_{11} + P_{12} + P_2 = 1
\end{equation}

Επιλύοντας το σύστημα προκύπτουν οι εξής εργοδικές πιθανότητες: 

\begin{center}
    \begin{tabular}{|c|c|c|c|c|}
        \hline
        $P_0$ & $P_{11}$ & $P_{12}$ & $P_2$ & $P_{blocking}$ \\
        \hline
        0.1387 & 0.1435 & 0.2457 & 0.4719 & 0.4719 \\
        \hline
    \end{tabular}
\end{center}

Η πιθανότητα απόρριψης είναι 0.4719. Ο μέσος αριθμός πελάτων στο σύστημα είναι ίσος με:

\begin{equation}
    L = \sum_{k = 0}^{2} kP_k = P_{11} + P_{12} + 2P_2 = 0.1435 + 0.2457 + 2*0.4719 = 1.333
\end{equation}






\subsection{Ερώτημα 2}

Για την συμπλήρωση των thresholds έχουμε:

\begin{enumerate}
    \item{}
        Το πρώτο threshold\_1a είναι η πιθανότητα να έχουμε άφιξη ενώ 
        ο πρώτος εξυπηρετητής, ο 1 σε αυτή την περίπτωση, είναι κατειλημμένος.

        \begin{equation}
            threshold\_1a = \frac{\lambda}{\lambda + \mu_1}
        \end{equation}
    \item{}
        Το δεύτερο threshold\_1b είναι η πιθανότητα να έχουμε άφιξη ενώ 
        ο δεύτερος εξυπηρετητής είναι κατειλημμένος.

        \begin{equation}
            threshold\_1b = \frac{\lambda}{\lambda + \mu_2}
        \end{equation}
    \item{}
        Το τρίτο threshold\_2\_first είναι η πιθανότητα να έχουμε άφιξη ενώ 
        και οι δύο εξυπηρετητές είναι κατειλημμένοι.

        \begin{equation}
            threshold\_2a = \frac{\lambda}{\lambda + \mu_1 + \mu_2}
        \end{equation}
    \item{}
        Το τέταρτο threshold\_2\_second είναι η πιθανότητα να έχουμε άφιξη ή 
        ο πρώτος εξυπηρετητής να απελευθερωθεί. Σε περίπτωση που υπερβούμε 
        αυτό το threshold τότε ο δεύτερος εξυπηρετητής απελευθερώνεται.

        \begin{equation}
            threshold\_2b = \frac{\lambda + \mu_1}{\lambda + \mu_1 + \mu_2}
        \end{equation}
        
\end{enumerate}
\\

Η συνθήκη σύγκλισης είναι μεταξύ δύο διαδοχικών μετρήσεων του μέσου αριθμού πελατών 
να έχουμε απόκλιση μικρότερη του $0.00001$ και φαίνεται στον κώδικα στην σειρά 
$38$.\\

Οι εργοδικές πιθανότητες που υπολογίζει η προσομοίωση φαίνονται παρακάτω και είναι 
σύμφωνες με τις πιθανότητες που υπολογίσαμε με μικρές αποκλίσεις.\\

\begin{center}
    \begin{tabular}{|c|c|c|c|c|}
        \hline
        $P_0$ & $P_{11}$ & $P_{12}$ & $P_2$ & $P_{blocking}$ \\
        \hline
        0.1392 & 0.1442 & 0.2442 & 0.4724 & 0.4724 \\
        \hline
    \end{tabular}
\end{center}

Ο κώδικας που χρησιμοποιήθηκε για τα παραπάνω είναι ο εξής:

\lstinputlisting[language=matlab, basicstyle=\tiny,
frame=single, backgroundcolor = \color{lightgray}]{../Code/ask2.m}


\end{document}
