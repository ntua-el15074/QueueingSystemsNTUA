\documentclass[12pt]{article}
\usepackage[greek,english]{babel}
\usepackage{alphabeta}
\usepackage{listings}
\usepackage{xcolor}
\usepackage{hyperref}
\usepackage{tabularx}
\usepackage{mathtools}
\usepackage{graphicx}
\usepackage{blindtext}
\usepackage{geometry}
\usepackage{listings}
\usepackage{amsmath}
\usepackage{amsfonts}
\usepackage{steinmetz}
\usepackage{algorithm}
\usepackage[noend]{algpseudocode}
\usepackage[shortlabels]{enumitem}
\usepackage{tikz}
\usepackage{fdsymbol}
\geometry{
    a4paper,
    total={170mm,257mm},
    left=20mm,
    top=20mm,
}

\author{Αυγερινός Πέτρος 03115074}
\title{Συστήματα Αναμονής Άσκηση 5^η}
\date{}

\begin{document}
\maketitle 
\pagebreak

\tableofcontents
\pagebreak

\section{Δίκτυο με εναλλακτική δρομολόγηση}
\subsection{Ερώτημα 1}
Για να μπορεί το σύστημα να μοντελοποιηθεί ως ουρά M/M/1 θα πρέπει τα $\rho_i$ να 
είναι μικρότερα της μονάδας.\\

Για την γραμμή 1 έχουμε ότι: 
\begin{equation}
    \lambda_1 = a \cdot 10^4 pps \text{ και } \mu_1 = \frac{c_1}{packet} = \frac{15 \cdot 10^6}{64 \cdot 8} = 29.296 \cdot 10^3 pps
\end{equation}

Για την γραμμή 2 έχουμε ότι: 
\begin{equation}
    \lambda_2 = (1-a) \cdot 10^4 pps \text{ και } \mu_2 = \frac{c_2}{packet} = \frac{12 \cdot 10^6}{64 \cdot 8} = 23.437 \cdot 10^3 pps
\end{equation}

Άρα θα πρέπει να ισχύει: 
\begin{equation}
    \rho_1 = \frac{\lambda_1}{\mu_1} < 1 \text{ και } \rho_2 = \frac{\lambda_2}{\mu_2} < 1
\end{equation}
\\

Για το $\rho_1$ έχουμε:
\begin{equation}
    \rho_1 = \frac{a \cdot 10^4}{29.296 \cdot 10^3} < 1 \Rightarrow a < 2.9296
\end{equation}

Για το $\rho_2$ έχουμε:
\begin{equation}
    \rho_2 = \frac{(1-a) \cdot 10^4}{23.437 \cdot 10^3} < 1 \Rightarrow 1-a < 2.3437 \Rightarrow a > -1.3437
\end{equation}

Όμως οι (4) και (5) ισχύουν καθώς το α είναι ποσοστό, και παίρνει τιμές μεταξύ 0 και 1. Επομένως το σύστημα είναι 
εργοδικό και το σύστημα μπορεί να μοντελοποιηθεί ως ουρά M/M/1.

\subsection{Ερώτημα 2}
Ο μέσος αριθμός πελατών στο σύστημα είναι: 
\begin{equation}
    Ε[N] = E[N_1] + E[N_2]
\end{equation}

Και γνωρίζουμε από τον τύπο του Little ότι ο μέσος χρόνος καθυστέρησης είναι: 
\begin{equation}
    Ε(Τ) = \frac{E[n]}{\lambda}
\end{equation}

Οι τιμές για την ελάχιστη καθυστέρηση για συγκεκριμένο α είναι:\\
\begin{center}
$Delay=4.60382e-05 \text{ για } a=0.674$
\end{center}
Παρακάτω φαίνεται η γραφική παράσταση της μέσης καθυστέρησης για τιμές του α από 0 έως 1.
\begin{figure}[H]
    \centering
    \includegraphics[width=0.8\textwidth]{../images/435434709_954432569486281_7935236155764955062_n.png}
    \caption{Μέση καθυστέρηση για τιμές του α}
\end{figure}

Ο κώδικας που χρησιμοποιηθήκε για τα παραπάνω είναι ο εξής:
\lstinputlisting[language=matlab, basicstyle=\tiny,
frame=single, backgroundcolor = \color{lightgray}]{../Code/ask1.m}
\pagebreak

\section{Ανοιχτό δίκτυο ουρών αναμονής}
\subsection{Ερώτημα 1}
Οι παραδοχές για να μπορεί να μελετηθεί το συγκεκριμένο σύστημα ως ανοιχτό δίκτυο 
με το θεώρημα Jackson είναι: 
\begin{itemize}
    \item{Τα $Q_i$ συνιστούν δικτυακούς κόμβους εξυπηρέτησης κορμού με εκθετικούς ρυθμούς 
        $\mu_i$}
    \item{Οι αφίξεις των πελατών προέρχονται από εξωτερικές πηγές που είναι άμεσα συνδεδεμένες 
        στους δικτυακούς κόμβους κορμού $Q_1$ και $Q_2$ και καταλήγουν στον κόμβο $Q_5$ με 
        απώλειες κατά τη διαδρομή. Οι ροές μεταξύ των δικτυακών κόμβων είναι ανεξάρτητες 
        ροές Poisson με μέσο ρυθμό $\gamma_{i,j}$ και η συνολική εξωγενής ροή Poisson είναι 
    ίση με $\gamma_i = \sum_{j=1}^{5} \gamma_{i,j}$ με $i \neq j$}
    \item{Η εσωτερική δρομολόγηση μεταξύ των κόμβων(ουρών) γίνεται με τυχαίο τρόπο από 
            και πιθανότητα $r_{i,j}$}
    \item{Οι ροές που διαπερνούν το κόμβο $Q_i$ έχουν συνολικό μέσο ρυθμό 
        $\lambda_j = \gamma_j + \sum_{i=1}^{5} r_{i,j}  \lambda_i$}
    \item{Οι χρόνοι εξυπηρέτησεις όπως διαπερνούν το δίκτυο δεν διατηρούν την τιμή τους, 
        είναι δηλαδή χωρίς μνήμη. Η τιμή τους εξαρτάται από την κατανομή του κάθε εξυπηρετητή.}
\end{itemize}

Τα $r_{i,j}$ είναι τα εξής: 
\begin{itemize}
    \item{$r_{1,2} = \dfrac{1}{2}$}
    \item{$r_{1,3} = \dfrac{1}{4}$}
    \item{$r_{2,3} = \dfrac{1}{3}$}
    \item{$r_{2,4} = \dfrac{2}{3}$}
    \item{$r_{4,5} = \dfrac{3}{5}$}
\end{itemize}

Τα $\rho_i$ είναι τα εξής:
\begin{itemize}
    \item{$\rho_1 = \dfrac{\lambda_1}{\mu_1}$}
    \item{$\rho_2 = \dfrac{\lambda_2 + r_{1,2}\cdot \lambda_1}{\mu_2}$}
    \item{$\rho_3 = \dfrac{r_{2,3}\cdot r_{1,2}\cdot \lambda_1 + r_{1,3} \cdot \lambda_1 + r_{2,3} \cdot \lambda_2}{\mu_3}$}
    \item{$\rho_4 = \dfrac{\lambda_2 \cdot r_{2,4}}{\mu_4}$}
    \item{$\rho_5 = \dfrac{\lambda_1 \cdot r_{1,3} \cdot r_{3,5} + \lambda_2 \cdot r_{2,4} \cdot r_{4,5}}{\mu_5}$}
\end{itemize}

\subsection{Ερώτημα 2}

Ο κώδικας της συνάρτησης intesities φαίνεται παρακάτω: 
\lstinputlisting[language=matlab, basicstyle=\tiny,
frame=single, backgroundcolor = \color{lightgray}]{../Code/intensities.m}

\subsection{Ερώτημα 3}

Ο κώδικας της συνάρτησης mean\_clients φαίνεται παρακάτω: 
\lstinputlisting[language=matlab, basicstyle=\tiny,
frame=single, backgroundcolor = \color{lightgray}]{../Code/mean_clients.m}

\subsection{Ερώτημα 4}

Οι ροές και η μέση καθυστέρηση για το συγκεκριμένο σύστημα είναι:
\lstinputlisting[language=matlab,
frame=single]{../Code/output.txt}

\pagebreak

Ο κώδικας για τα παραπάνω αποτελέσματα δίνεται παρακάτω:
\lstinputlisting[language=matlab, basicstyle=\tiny,
frame=single, backgroundcolor = \color{lightgray}]{../Code/ask2_4.m}

\subsection{Ερώτημα 5}

Ο στενωπός είναι η ουρά 1 καθώς έχει τη μεγαλύτερη ροή σύμφωνα με τα αποτελέσματα του 
παραπάνω ερωτήματος.
\\

Για να μεγιστοποιήσουμε το $\lambda_1$ ενώ παράλληλα διατηρούμε τα $\rho_i$ μικρότερα της μονάδας
θα πρέπει να λύσουμε το σύστημα των ανισώσεων που προκύπτουν από τις παραπάνω σχέσεις.
Η λύση του συστήματος είναι προφανής και είναι ίση με το $\lambda_1 = 15$ η οποία προκύπτει 
απο τη $\rho_1$ καθώς όλες οι υπόλοιπες ροές προκαλούν μεγαλύτερα αποτελέσματα στο άνω άκρο 
των ανισώσεων.

\subsection{Ερώτημα 6}
Το διάγραμμα που απεικονίζει τη μέση καθυστέρηση για $\lambda_1$ από 1,5 μέχρι 14,85 με βήμα 0,15 
φαίνεται παρακάτω: 
\begin{figure}[H]
    \centering
    \includegraphics[width=0.8\textwidth]{../images/delay_to_lambda.png}
    \caption{Μέση καθυστέρηση για τιμές του $\lambda_1$}
\end{figure}

Ενώ ο κώδικας για την πραγματοποίηση του παραπάνω διαγράμματος είναι ο εξής:
\lstinputlisting[language=matlab, basicstyle=\tiny,
frame=single, backgroundcolor = \color{lightgray}]{../Code/ask2_6.m}

    

\end{document}
