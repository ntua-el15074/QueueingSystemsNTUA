\documentclass[12pt]{article}
\usepackage[greek,english]{babel}
\usepackage{alphabeta}
\usepackage{listings}
\usepackage{xcolor}
\usepackage[backend=biber]{biblatex}
\usepackage{hyperref}
\usepackage{tabularx}
\usepackage{mathtools}
\usepackage{graphicx}
\usepackage{blindtext}
\usepackage{geometry}
\usepackage{listings}
\usepackage{amsmath}
\usepackage{amsfonts}
\usepackage{steinmetz}
\usepackage{algorithm}
\usepackage[noend]{algpseudocode}
\usepackage[shortlabels]{enumitem}
\usepackage{tikz}
\usepackage{fdsymbol}
\geometry{
    a4paper,
    total={170mm,257mm},
    left=20mm,
    top=20mm,
}

\author{Αυγερινός Πέτρος 03115074}
\title{Συστήματα Αναμονής Άσκηση 2^η}
\date{}

\begin{document}
\maketitle 
\pagebreak

\tableofcontents


\pagebreak

\section{Θεωρητική Μελέτη της ουράς Μ/Μ/1}

\subsection{Ερώτημα Α}
Η απαραίτητη συνθήκη για μία ουρά Μ/Μ/1 να είναι εργοδική είναι η ένταση
κυκλοφορίας $\rho = \frac{\lambda}{\mu} < 1$. Η ένταση κυκλοφορίας εκφράζει
την πιθανότητα να είναι απασχολημένη η μονάδα εξυπηρέτησης.

Γνωρίζουμε ότι οι εξισώσεις ισορροπίας είναι: 
\begin{equation}
    (\lambda_k + \mu_k)P_k = \lambda_{k-1}P_{k-1} + \mu_{k+1}P_{k+1},  k \ge 1
\end{equation}

και 

\begin{equation}
    \lambda_0P_0 = \mu_1P_1
\end{equation}

Για την ουρά Μ/Μ/1 ισχύει ότι $\lambda_k = \lambda, \mu_k = \mu, \forall k \ge 1$.
Επομένως οι εξισώσεις ισορροπίας γίνονται:
\begin{equation}
    \lambda P_0 = \mu P_1 \Rightarrow P_1 = \frac{\lambda}{\mu}P_0 = \rho P_0
\end{equation}

και

\begin{equation}
    P_k = \rho P_{k-1} = \rho^k P_0, k > 0
\end{equation}

Όμως η άθροιση των πιθανοτήτων πρέπει να είναι 1, δηλαδή:
\begin{equation}
    \sum_{k=0}^{\infty} P_k = 1 \Rightarrow P_0 \sum_{k=0}^{\infty} \rho^k = 1 \Rightarrow P_0 \frac{1}{1-\rho} = 1 \Rightarrow P_0 = 1 - \rho
\end{equation}

Επομένως προκύπτει η πιθανότητα $P_k$:
\begin{equation}
    P_k = (1-\rho)\rho^k
\end{equation}

Το διάγραμμα ρυθμού μεταβάσεων της ουράς Μ/Μ/1 φαίνεται στο Σχήμα: 

\begin{figure}[H]
    \centering
    \includegraphics[width=0.5\textwidth]{../Images/task1ask1.png}
    \caption{Διάγραμμα Ρυθμού Μεταβάσεων Μ/Μ/1}
    \label{fig:task1ask1}
\end{figure}

\subsection{Ερώτημα Β}

Για τον υπολογισμό του μέσου χρόνου καθυστέρησης στην ουρά Μ/Μ/1 χρησιμοποιούμε τον τύπο:
\begin{equation}
    Ε(Τ) = \frac{1}{\mu} \cdot \frac{1}{1-\rho}
\end{equation}

Ο μέσος χρόνος αναμονής είναι ίσος με:
\begin{equation}
    Ε(W) = Ε(T) - \frac{1}{\mu} = \frac{1}{\mu} \cdot \frac{\rho}{1-\rho}
\end{equation}

\subsection{Ερώτημα Γ}
Για να υπολογίσουμε την πιθανότητα να υπάρχουν 3 τουλάχιστον πελάτες στο σύστημα θα 
εργαστούμε ως εξής: 

\begin{equation}
    P(N \ge 3) = \sum_{k=3}^{\infty} P_k = \sum_{k=3}^{\infty} (1-\rho)\rho^k
\end{equation}

Αυτή η πιθανότητα μπορεί να υπολογιστεί ως η εξής συμπληρωματική πιθανότητα: 

\begin{equation}
    P(N \ge 3) = 1 - P(N < 3) = 1 - P(N = 0) - P(N = 1) - P(N = 2)
\end{equation}

Για ένα δεδομένο ρ όπου $\rho < 1$.

\subsection{Ερώτημα Δ}
Για να υπολογίσουμε την πιθανότητα να βρεθεί το σύστημα με 107 πελάτες στην ουρά θα 
χρησιμοποιήσουμε την παρακάτω εξίσωση:

\begin{equation}
    P_{107} = (1-\rho)\rho^{107}
\end{equation}

Για τιμές του ρ όπου $\rho \le 1$ οι πιθανότητες είναι πολύ μικρές. Βέβαια όσο μεγαλώνει το 
$\rho$ τόσο περισσότερο αυξάνεται και η πιθανότητα αυτή. Σαφώς για μικρά $\rho$ η πιθανότητα 
είναι αμελητέα.

Για παράδειγμα:
\begin{enumerate}
    \item{Για $\rho = 0.1$ η πιθανότητα είναι $9.000000000000054e-108$.}
    \item{Για $\rho = 0.5$ η πιθανότητα είναι $3.0814879110195774e-33$.}
    \item{Για $\rho = 0.9$ η πιθανότητα είναι $1.270423474759657e-06$.}
\end{enumerate}

\pagebreak

\section{Ανάλυση ουράς Μ/Μ/1 με Octave}

\subsection{Ερώτημα Α}

Για να είναι το σύστημα εργοδικό θα πρέπει να ισχύει ότι $\rho = \frac{\lambda}{\mu} < 1$.
Γνωρίζουμε ότι το $\lambda = 10 \text{πελάτες/min}$ ενώ το $\mu$ εκτείνεται σε εύρος τιμών 
από 0 έως 20 πελάτες ανά λεπτό. Επομένως αποδεκτές τιμές του $\mu$ είναι από 10 έως 20.

Άρα θα πρεπει $\mu > 10$.

\subsection{Ερώτημα Β}

Θα κάνουμε χρήση της συνάρτησης qsmm1 του Octave για να λάβουμε όλα τα απαραίτητα στοιχεία 
για την ουρά μας.

Ο κώδικας αρχικοποίησης της ουράς είναι ο εξής:

\begin{lstlisting}[language=C,
                   frame=single,
                   numberstyle=\color{codegray},
                   basicstyle=\footnotesize,
                   numbers=left,
                   backgroundcolor=\color{lightgray},
                   numbersep=5pt]
pkg load queueing;

lambda = 10;

mu = 10.0001 : 0.0001 : 20;

[U, R, Q, X, p0] = qsmm1(lambda, mu);

colors = "rgbm";
\end{lstlisting}

\pagebreak

Για κάθε ένα από τα ζητούμενα της άσκησης έχουμε:
\begin{enumerate}
    \item{}
        Σχετικά με το utilisation, το διάγραμμα που προκύπτει είναι το εξής:
        \begin{center}
            \includegraphics[scale=0.6]{../Images/utilisation.jpg}
        \end{center}

        Ο κώδικας που χρησιμοποιήθηκε για το παραπάνω διάγραμμα είναι ο εξής:
            \begin{lstlisting}[language=C,
                               frame=single,
                               numberstyle=\color{codegray},
                               basicstyle=\footnotesize,
                               numbers=left,
                               backgroundcolor=\color{lightgray},
                               numbersep=5pt]
figure(1); 
plot(mu,U,colors(1),'linewidth',1.2);
xlabel('Service Rate');
ylabel('Utilisation');
title('Utilisation');

print('-djpg', 'utilisation.jpg');
            \end{lstlisting}

    \pagebreak
    \item{}
        Σχετικά με το μέσο χρόνο καθυστέρησης, το διάγραμμα που προκύπτει είναι το εξής:
        \begin{center}
            \includegraphics[scale=0.6]{../Images/responsetime.jpg}
        \end{center}

        Ο κώδικας που χρησιμοποιήθηκε για το παραπάνω διάγραμμα είναι ο εξής:
            \begin{lstlisting}[language=C,
                               frame=single,
                               numberstyle=\color{codegray},
                               basicstyle=\footnotesize,
                               numbers=left,
                               backgroundcolor=\color{lightgray},
                               numbersep=5pt]
figure(2);
plot(mu,R,colors(1),'linewidth',1.2);
ylim([0,100]);
xlabel('Service Rate');
ylabel('Response Time');
title('Response Time');

print('-djpg', 'responsetime.jpg');
            \end{lstlisting}

    \pagebreak
    \item{}
        Σχετικά με το μέσο αριθμό των πελατών, το διάγραμμα που προκύπτει είναι το εξής:
        \begin{center}
            \includegraphics[scale=0.6]{../Images/average_n_requests.jpg}
        \end{center}

        Ο κώδικας που χρησιμοποιήθηκε για το παραπάνω διάγραμμα είναι ο εξής:
            \begin{lstlisting}[language=C,
                               frame=single,
                               numberstyle=\color{codegray},
                               basicstyle=\footnotesize,
                               numbers=left,
                               backgroundcolor=\color{lightgray},
                               numbersep=5pt]
figure(3); 
plot(mu,Q,colors(1),'linewidth',1.2);
ylim([0,100]);
xlabel('Service Rate');
ylabel('Average number of requests');

print('-djpg', 'average_n_requests.jpg');
            \end{lstlisting}

    \pagebreak
    \item{}
        Σχετικά με τη ρυθμαπόδοση, το διάγραμμα που προκύπτει είναι το εξής:
        \begin{center}
            \includegraphics[scale=0.6]{../Images/throughput.jpg}
        \end{center}

        Ο κώδικας που χρησιμοποιήθηκε για το παραπάνω διάγραμμα είναι ο εξής:
            \begin{lstlisting}[language=C,
                               frame=single,
                               numberstyle=\color{codegray},
                               basicstyle=\footnotesize,
                               numbers=left,
                               backgroundcolor=\color{lightgray},
                               numbersep=5pt]
figure(4);
plot(mu,X,colors(1), 'linewidth',1.2);
xlabel('Service Rate');
ylabel('Throughput');

print('-djpg', 'throughput.jpg');
            \end{lstlisting}


\end{enumerate}

\subsection{Ερώτημα Γ}
Ο μέσος χρόνος εξυπηρέτησης είναι αντιστρόφος ανάλογος του ρυθμού εξυπηρέτησης, 
επομένως για μεγάλα $\mu$ ο μέσος χρόνος εξυπηρέτησης είναι μικρός. Η αύξηση 
του ρυθμού εξυπηρέτησης όμως απαιτεί σημαντική ποσότητα πόρων, επομένως η αύξηση 
του ρυθμού θα πρέπει να γίνει με τέτοιο τρόπο ώστε να μειωθεί ο χρόνος εξυπηρέτησεις 
αλλά να μην αυξηθούν σημαντικά οι ανάγκες της ουράς για πόρους. Επομένως από το 
διάγραμα θα επέλεγα το $\mu = 12$ για το οποίο υπάρχει ένας ανεπέστατος χρόνος 
εξυπηρέτησης. Αλλιώς μπορούμε να επιλέξουμε και την μέση λύση του $\mu = 15$.


\subsection{Ερώτημα Δ}
Το throughput στην ουρα παρατηρούμε ότι μένει σταθερό για όλες τις τιμές του $\mu$.
Αυτό συμβαίνει γιατί η ρυθμαπόδοση εξαρτάται από την παράμετρο $\lambda$ και την 
πιθανότητα απώλειας η οποία είναι 0 σε μία Μ/Μ/1 ουρά η οποία έχει άπειρη χωρητικότητα.

\pagebreak

\section{Διαδικασία γεννήσεων θανάτων (birth-death process): εφαρμογή σε σύστημα
Μ/Μ/1/Κ}

\subsection{Ερώτημα Α}

Αρχικά έχουμε τα εξής γνωστά:
\begin{enumerate}
    \item{Γνωρίζουμε ότι $\lambda = 5$ και $\mu = 10$}
    \item{Γνωρίζουμε επίσης ότι $\lambda_i = \frac{\lambda}{(i+2)}$ και $\mu_i = \mu, i = 0,1,2$}
\end{enumerate}

Από τις εξισώσεις ισορροπίας και την συνθήκη κανονικοποίησης έχουμε τις
εξής εργοδικές πιθανότητες για την ουρά Μ/Μ/1/3: 
\begin{equation}
    P_0 = \frac{1}{1 + \frac{\lambda_0}{\mu} + \frac{\lambda_0\lambda_1}{\mu^2} + \frac{\lambda_0\lambda_1\lambda_2}{\mu^3}} = 0.771084337
\end{equation}

\begin{equation}
    P_1 = \frac{\lambda_0}{\mu}P_0 = 0.19275
\end{equation}

\begin{equation}
    P_2 = \frac{\lambda_1}{\mu}P_1 = 0.032125
\end{equation}

\begin{equation}
    P_3 = \frac{\lambda_2}{\mu}P_2 = 0.004015625
\end{equation}

Γνωρίζουμε ότι $P_{blocking} = P_3$.\\


Το διάγραμμα γεννήσεων-θανάτων για μία ουρά Μ/Μ/1/3 φαίνεται στο παρακάτω σχήμα:

\begin{figure}[H]
    \centering
    \includegraphics[width=0.5\textwidth]{../Images/birth_death_diagram.png}
    \caption{Διάγραμμα Γεννήσεων-Θανάτων Μ/Μ/1/3}
    \label{fig:task3ask1}
\end{figure}


\pagebreak
\subsection{Ερώτημα Β}

\begin{enumerate}
    \item{}
        Η μήτρα ρυθμού μεταβάσεων φαίνεται παρακάτω:\\
        \begin{table}[h!]
          \centering
          \begin{tabular}{|c|c|c|c|}
            \hline
              -2.50000&   2.50000&   0.00000&   0.00000 \\
              10.00000& -11.66667&   1.66667&   0.00000 \\
               0.00000&  10.00000& -11.25000&   1.25000 \\
               0.00000&   0.00000&  10.00000& -10.00000 \\
            \hline
          \end{tabular}
        \end{table}

    \item{}
        Οι εργοδικές πιθανότητες φαίνονται παρακάτω στο διάγραμμα:\\
        \begin{center}
            \includegraphics[scale=0.6]{../Images/bar_plot.jpg}
        \end{center}

        Και σε συγκεκριμένες τιμές εδώ:\\
        \begin{table}[h!]
          \centering
          \begin{tabular}{|c|c|c|c|c|}
            \hline
            $P_0$ & $P_1$ & $P_2$ & $P_3$ & $P_{blocking}$ \\ \hline
            0.771084 & 0.192771 & 0.0321285 & 0.00401606 & 0.00401606 \\
            \hline
          \end{tabular}
        \end{table}

    \item{}
        Η πιθανότητα απόρριψης είναι $P_{blocking} = 0.00401606$.

    \item{}
        Ο μέσος αριθμός πελατών στην ουρά είναι $0.26908$ σύμφωνα με τον τύπο:
        \begin{equation}
            L = \sum_{i=0}^{3} i \cdot P_i
        \end{equation}

    \item{}
        Ο μέσος αριθμός εξυπηρετούμενων πελατών σε ένα διάστημα χρόνου ίσο με 60 δευτερόλεπτα είναι:
        \begin{equation}
            L_s = L \cdot \mu \cdot 60 = 161.448
        \end{equation}

    \item{}
        Παρακάτω φαίνονται τα διαγράμματα  των πιθανοτήτων καταστάσεων:
        \begin{enumerate}
            \item{}
                \begin{center}
                    \includegraphics[scale=0.4]{../Images/probability_0_5_10.jpg}
                \end{center}
            \item{}
                \begin{center}
                    \includegraphics[scale=0.4]{../Images/probability_1_5_10.jpg}
                \end{center}
            \item{}
                \begin{center}
                    \includegraphics[scale=0.4]{../Images/probability_2_5_10.jpg}
                \end{center}
            \item{}
                \begin{center}
                    \includegraphics[scale=0.4]{../Images/probability_3_5_10.jpg}
                \end{center}
        \end{enumerate}
Ο κώδικας που χρησιμοποιήθηκε για τα παραπάνω φαίνεται παρακάτω:

\begin{lstlisting}[language=C,
                   frame=single,
                   numberstyle=\color{codegray},
                   basicstyle=\footnotesize,
                   numbers=left,
                   backgroundcolor=\color{lightgray},
                   numbersep=5pt]
% system M/M/1/3

clc;
clear all;
close all;

pkg load queueing;

lambda = 5;
mu = 10;
states = [0, 1, 2, 3]; % system with capacity 4 states
% the initial state of the system. The system is initially empty.
initial_state = [1, 0, 0, 0];

% define the birth and death rates between the states of the system.
births_B = [lambda/2, lambda/3, lambda/4];
deaths_D = [mu, mu, mu];

% get the transition matrix of the birth-death process
transition_matrix = ctmcbd(births_B, deaths_D);
display('transition matrix');
display(transition_matrix);
% get the ergodic probabilities of the system
P = ctmc(transition_matrix);


% plot the ergodic probabilities (bar for bar chart)
if mu = 10
	figure(1);
	bar(states, P, "r", 0.5);
	xlabel('States');
	ylabel('Propabilities');
	
	print('-djpg','bar_plot.jpg');
endif

display('blocking probability');
display(P(4));

mean_clients = 0;
for i=1:4
	mean_clients = mean_clients + P(i)*(i-1);
	probtext = sprintf('probability %d: %d', i-1, P(i));
	display(probtext);
endfor
display('mean');
display(mean_clients);

for j=1:4
	index = 0;
	for T = 0 : 0.01 : 50
	  index = index + 1;
	  P0 = ctmc(transition_matrix, T, initial_state);
	  Prob0(index) = P0(j);
	  if P0 - P < 0.01
	    break;
	  endif
	endfor

	T = 0 : 0.01 : T;
	figure(2);
	plot(T, Prob0, "r", "linewidth", 1.3);
	plotname = sprintf('probability %d',j-1);
	title(plotname);
	xlabel('time');
	ylabel('probability');
	filename = sprintf('probability_%d_%d_%d.jpg',j-1, lambda, mu);

	print('-djpg', filename);
endfor

result = mean_clients * mu * 60;
text = sprintf('mean_clients(%d): %d', mu, result);
display(text);

\end{lstlisting}

\item{}
        \begin{enumerate}
            \item{}
                Για $mu = 1$, ο συνολικός αριθμός είναι 123.301 και έχουμε τα παρακάτω διαγράμματα:
                \begin{enumerate}
                    \item{}
                        \begin{center}
                            \includegraphics[scale=0.4]{../Images/probability_0_5_1.jpg}
                        \end{center}
                    \item{}
                        \begin{center}
                            \includegraphics[scale=0.4]{../Images/probability_1_5_1.jpg}
                        \end{center}
                    \item{}
                        \begin{center}
                            \includegraphics[scale=0.4]{../Images/probability_2_5_1.jpg}
                        \end{center}
                    \item{}
                        \begin{center}
                            \includegraphics[scale=0.4]{../Images/probability_3_5_1.jpg}
                        \end{center}
                \end{enumerate}

                \item{}
                    Για $mu = 5$ ο συνολικός αριθμός είναι 168.293 και έχουμε τα παρακάτω διαγράμματα:
                    \begin{enumerate}
                        \item{}
                            \begin{center}
                                \includegraphics[scale=0.4]{../Images/probability_0_5_5.jpg}
                            \end{center}
                        \item{}
                            \begin{center}
                                \includegraphics[scale=0.4]{../Images/probability_1_5_5.jpg}
                            \end{center}
                        \item{}
                            \begin{center}
                                \includegraphics[scale=0.4]{../Images/probability_2_5_5.jpg}
                            \end{center}
                        \item{}
                            \begin{center}
                                \includegraphics[scale=0.4]{../Images/probability_3_5_5.jpg}
                            \end{center}
                    \end{enumerate}

            \item{}
                Για $mu = 20$ ο συνολικός αριθμός είναι 156.103 και έχουμε τα παρακάτω διαγράμματα:
                \begin{enumerate}
                    \item{}
                        \begin{center}
                            \includegraphics[scale=0.4]{../Images/probability_0_5_20.jpg}
                        \end{center}
                    \item{}
                        \begin{center}
                            \includegraphics[scale=0.4]{../Images/probability_1_5_20.jpg}
                        \end{center}
                    \item{}
                        \begin{center}
                            \includegraphics[scale=0.4]{../Images/probability_2_5_20.jpg}
                        \end{center}
                    \item{}
                        \begin{center}
                            \includegraphics[scale=0.4]{../Images/probability_3_5_20.jpg}
                        \end{center}
                \end{enumerate}


        Δημιουργήθηκαν από το κώδικα του προηγούμενου ερωτήματος παραμετροποιώντας το $\mu$.
        Διατηρώντας σταθερό το $\lambda$ και αυξάνοντας το $\mu$ παρατηρούμε ότι συγκλίνουμε 
        στις εργοδικές πιθανότητε πιο γρήγορα. Αυτό συμβαίνει γιατί όσο αυξάνεται το $\mu$
        μειώνονται οι εργοδικές πιθανότητες.\\
        
        \pagebreak

\end{enumerate}





















\end{document}
