\documentclass[12pt]{article}
\usepackage[greek,english]{babel}
\usepackage{alphabeta}
\usepackage{listings}
\usepackage{xcolor}
\usepackage[backend=biber]{biblatex}
\usepackage{hyperref}
\usepackage{tabularx}
\usepackage{mathtools}
\usepackage{graphicx}
\usepackage{blindtext}
\usepackage{geometry}
\usepackage{listings}
\usepackage{amsmath}
\usepackage{amsfonts}
\usepackage{steinmetz}
\usepackage{algorithm}
\usepackage[noend]{algpseudocode}
\usepackage[shortlabels]{enumitem}
\usepackage{tikz}
\usepackage{fdsymbol}
\geometry{
    a4paper,
    total={170mm,257mm},
    left=20mm,
    top=20mm,
}

\author{Αυγερινός Πέτρος 03115074}
\title{Συστήματα Αναμονής Άσκηση 2^η}
\date{}

\begin{document}
\maketitle 
\pagebreak

\tableofcontents


\pagebreak

\section{Θεωρητική Μελέτη της ουράς Μ/Μ/1}

\subsection{Ερώτημα Α}
Η απαραίτητη συνθήκη για μία ουρά Μ/Μ/1 να είναι εργοδική είναι η ένταση
κυκλοφορίας $\rho = \frac{\lambda}{\mu} < 1$. Η ένταση κυκλοφορίας εκφράζει
την πιθανότητα να είναι απασχολημένη η μονάδα εξυπηρέτησης.

Γνωρίζουμε ότι οι εξισώσεις ισορροπίας είναι: 
\begin{equation}
    (\lambda_k + \mu_k)P_k = \lambda_{k-1}P_{k-1} + \mu_{k+1}P_{k+1},  k \ge 1
\end{equation}

και 

\begin{equation}
    \lambda_0P_0 = \mu_1P_1
\end{equation}

Για την ουρά Μ/Μ/1 ισχύει ότι $\lambda_k = \lambda, \mu_k = \mu, \forall k \ge 1$.
Επομένως οι εξισώσεις ισορροπίας γίνονται:
\begin{equation}
    \lambda P_0 = \mu P_1 \Rightarrow P_1 = \frac{\lambda}{\mu}P_0 = \rho P_0
\end{equation}

και

\begin{equation}
    P_k = \rho P_{k-1} = \rho^k P_0, k > 0
\end{equation}

Όμως η άθροιση των πιθανοτήτων πρέπει να είναι 1, δηλαδή:
\begin{equation}
    \sum_{k=0}^{\infty} P_k = 1 \Rightarrow P_0 \sum_{k=0}^{\infty} \rho^k = 1 \Rightarrow P_0 \frac{1}{1-\rho} = 1 \Rightarrow P_0 = 1 - \rho
\end{equation}

Επομένως προκύπτει η πιθανότητα $P_k$:
\begin{equation}
    P_k = (1-\rho)\rho^k
\end{equation}

Το διάγραμμα ρυθμού μεταβάσεων της ουράς Μ/Μ/1 φαίνεται στο Σχήμα: 

\begin{figure}[H]
    \centering
    \includegraphics[width=0.5\textwidth]{../Images/task1ask1.png}
    \caption{Διάγραμμα Ρυθμού Μεταβάσεων Μ/Μ/1}
    \label{fig:task1ask1}
\end{figure}

\subsection{Ερώτημα Β}

Για τον υπολογισμό του μέσου χρόνου καθυστέρησης στην ουρά Μ/Μ/1 χρησιμοποιούμε τον τύπο:
\begin{equation}
    Ε(Τ) = \frac{1}{\mu} \cdot \frac{1}{1-\rho}
\end{equation}

Ο μέσος χρόνος αναμονής είναι ίσος με:
\begin{equation}
    Ε(W) = Ε(T) - \frac{1}{\mu} = \frac{1}{\mu} \cdot \frac{\rho}{1-\rho}
\end{equation}

\subsection{Ερώτημα Γ}
Για να υπολογίσουμε την πιθανότητα να υπάρχουν 3 τουλάχιστον πελάτες στο σύστημα θα 
εργαστούμε ως εξής: 

\begin{equation}
    P(N \ge 3) = \sum_{k=3}^{\infty} P_k = \sum_{k=3}^{\infty} (1-\rho)\rho^k
\end{equation}

Αυτή η πιθανότητα μπορεί να υπολογιστεί ως η εξής συμπληρωματική πιθανότητα: 

\begin{equation}
    P(N \ge 3) = 1 - P(N < 3) = 1 - P(N = 0) - P(N = 1) - P(N = 2)
\end{equation}

Για ένα δεδομένο ρ όπου $\rho < 1$.

\subsection{Ερώτημα Δ}
Για να υπολογίσουμε την πιθανότητα να βρεθεί το σύστημα με 107 πελάτες στην ουρά θα 
χρησιμοποιήσουμε την παρακάτω εξίσωση:

\begin{equation}
    P_{107} = (1-\rho)\rho^{107}
\end{equation}

Για τιμές του ρ όπου $\rho \le 1$ οι πιθανότητες είναι πολύ μικρές. Βέβαια όσο μεγαλώνει το 
$\rho$ τόσο περισσότερο αυξάνεται και η πιθανότητα αυτή. Σαφώς για μικρά $\rho$ η πιθανότητα 
είναι αμελητέα.

Για παράδειγμα:
\begin{enumerate}
    \item{Για $\rho = 0.1$ η πιθανότητα είναι $9.000000000000054e-108$.}
    \item{Για $\rho = 0.5$ η πιθανότητα είναι $3.0814879110195774e-33$.}
    \item{Για $\rho = 0.9$ η πιθανότητα είναι $1.270423474759657e-06$.}
\end{enumerate}

\pagebreak

\section{Ανάλυση ουράς Μ/Μ/1 με Octave}

\subsection{Ερώτημα Α}

Για να είναι το σύστημα εργοδικό θα πρέπει να ισχύει ότι $\rho = \frac{\lambda}{\mu} < 1$.
Γνωρίζουμε ότι το $\lambda = 10 \text{πελάτες/min}$ ενώ το $\mu$ εκτείνεται σε εύρος τιμών 
από 0 έως 20 πελάτες ανά λεπτό. Επομένως αποδεκτές τιμές του $\mu$ είναι από 10 έως 20.

Άρα θα πρεπει $\mu > 10$.

\subsection{Ερώτημα Β}

Θα κάνουμε χρήση της συνάρτησης qsmm1 του Octave για να λάβουμε όλα τα απαραίτητα στοιχεία 
για την ουρά μας.

Ο κώδικας αρχικοποίησης της ουράς είναι ο εξής:

\begin{lstlisting}[language=C,
                   frame=single,
                   numberstyle=\color{codegray},
                   basicstyle=\footnotesize,
                   numbers=left,
                   backgroundcolor=\color{lightgray},
                   numbersep=5pt]
pkg load queueing;

lambda = 10;

mu = 10.0001 : 0.0001 : 20;

[U, R, Q, X, p0] = qsmm1(lambda, mu);

colors = "rgbm";
\end{lstlisting}

\pagebreak

Για κάθε ένα από τα ζητούμενα της άσκησης έχουμε:
\begin{enumerate}
    \item{}
        Σχετικά με το utilisation, το διάγραμμα που προκύπτει είναι το εξής:
        \begin{center}
            \includegraphics[scale=0.6]{../Images/utilisation.jpg}
        \end{center}

        Ο κώδικας που χρησιμοποιήθηκε για το παραπάνω διάγραμμα είναι ο εξής:
            \begin{lstlisting}[language=C,
                               frame=single,
                               numberstyle=\color{codegray},
                               basicstyle=\footnotesize,
                               numbers=left,
                               backgroundcolor=\color{lightgray},
                               numbersep=5pt]
figure(1); 
plot(mu,U,colors(1),'linewidth',1.2);
xlabel('Service Rate');
ylabel('Utilisation');
title('Utilisation');

print('-djpg', 'utilisation.jpg');
            \end{lstlisting}

    \pagebreak
    \item{}
        Σχετικά με το μέσο χρόνο καθυστέρησης, το διάγραμμα που προκύπτει είναι το εξής:
        \begin{center}
            \includegraphics[scale=0.6]{../Images/responsetime.jpg}
        \end{center}

        Ο κώδικας που χρησιμοποιήθηκε για το παραπάνω διάγραμμα είναι ο εξής:
            \begin{lstlisting}[language=C,
                               frame=single,
                               numberstyle=\color{codegray},
                               basicstyle=\footnotesize,
                               numbers=left,
                               backgroundcolor=\color{lightgray},
                               numbersep=5pt]
figure(2);
plot(mu,R,colors(1),'linewidth',1.2);
ylim([0,100]);
xlabel('Service Rate');
ylabel('Response Time');
title('Response Time');

print('-djpg', 'responsetime.jpg');
            \end{lstlisting}

    \pagebreak
    \item{}
        Σχετικά με το μέσο αριθμό των πελατών, το διάγραμμα που προκύπτει είναι το εξής:
        \begin{center}
            \includegraphics[scale=0.6]{../Images/average_n_requests.jpg}
        \end{center}

        Ο κώδικας που χρησιμοποιήθηκε για το παραπάνω διάγραμμα είναι ο εξής:
            \begin{lstlisting}[language=C,
                               frame=single,
                               numberstyle=\color{codegray},
                               basicstyle=\footnotesize,
                               numbers=left,
                               backgroundcolor=\color{lightgray},
                               numbersep=5pt]
figure(3); 
plot(mu,Q,colors(1),'linewidth',1.2);
ylim([0,100]);
xlabel('Service Rate');
ylabel('Average number of requests');

print('-djpg', 'average_n_requests.jpg');
            \end{lstlisting}

    \pagebreak
    \item{}
        Σχετικά με τη ρυθμαπόδοση, το διάγραμμα που προκύπτει είναι το εξής:
        \begin{center}
            \includegraphics[scale=0.6]{../Images/throughput.jpg}
        \end{center}

        Ο κώδικας που χρησιμοποιήθηκε για το παραπάνω διάγραμμα είναι ο εξής:
            \begin{lstlisting}[language=C,
                               frame=single,
                               numberstyle=\color{codegray},
                               basicstyle=\footnotesize,
                               numbers=left,
                               backgroundcolor=\color{lightgray},
                               numbersep=5pt]
figure(4);
plot(mu,X,colors(1), 'linewidth',1.2);
xlabel('Service Rate');
ylabel('Throughput');

print('-djpg', 'throughput.jpg');
            \end{lstlisting}


\end{enumerate}

\subsection{Ερώτημα Γ}
Ο μέσος χρόνος εξυπηρέτησης είναι αντιστρόφος ανάλογος του ρυθμού εξυπηρέτησης, 
επομένως για μεγάλα $\mu$ ο μέσος χρόνος εξυπηρέτησης είναι μικρός. Η αύξηση 
του ρυθμού εξυπηρέτησης όμως απαιτεί σημαντική ποσότητα πόρων, επομένως η αύξηση 
του ρυθμού θα πρέπει να γίνει με τέτοιο τρόπο ώστε να μειωθεί ο χρόνος εξυπηρέτησεις 
αλλά να μην αυξηθούν σημαντικά οι ανάγκες της ουράς για πόρους. Επομένως από το 
διάγραμα θα επέλεγα το $\mu = 12$ για το οποίο υπάρχει ένας ανεπέστατος χρόνος 
εξυπηρέτησης. Αλλιώς μπορούμε να επιλέξουμε και την μέση λύση του $\mu = 15$.


\subsection{Ερώτημα Δ}
Το throughput στην ουρα παρατηρούμε ότι μένει σταθερό για όλες τις τιμές του $\mu$.
Αυτό συμβαίνει γιατί η ρυθμαπόδοση εξαρτάται από την παράμετρο $\lambda$ και την 
πιθανότητα απώλειας η οποία είναι 0 σε μία Μ/Μ/1 ουρά η οποία έχει άπειρη χωρητικότητα.

\pagebreak

\section{Διαδικασία γεννήσεων θανάτων (birth-death process): εφαρμογή σε σύστημα
Μ/Μ/1/Κ}





















\end{document}
