\documentclass[12pt]{article}
\usepackage[greek,english]{babel}
\usepackage{alphabeta}
\usepackage{listings}
\usepackage{xcolor}
\usepackage[backend=biber]{biblatex}
\usepackage{hyperref}
\usepackage{tabularx}
\usepackage{mathtools}
\usepackage{graphicx}
\usepackage{blindtext}
\usepackage{geometry}
\usepackage{listings}
\usepackage{amsmath}
\usepackage{amsfonts}
\usepackage{steinmetz}
\usepackage{algorithm}
\usepackage[noend]{algpseudocode}
\usepackage[shortlabels]{enumitem}
\usepackage{tikz}
\usepackage{fdsymbol}
\geometry{
    a4paper,
    total={170mm,257mm},
    left=20mm,
    top=20mm,
}

\lstdefinestyle{mystyle}{
    commentstyle=\color{codegreen},
    keywordstyle=\color{magenta},
    numberstyle=\tiny\color{codegray},
    stringstyle=\color{codepurple},
    basicstyle=\tiny,
    frame=single,
}

\lstset{style=mystyle}

\author{Αυγερινός Πέτρος 03115074}
\title{Συστήματα Αναμονής Άσκηση 1^η}
\date{}

\begin{document}
\maketitle 
\pagebreak

\tableofcontents

\pagebreak

\section{Κατανομή Poisson}

\subsection{Ερώτημα A}
Η διακριτή τυχαία μεταβλητή ακολουθεί κατανομή Poisson με παράμετρο 
$\lambda > 0$. Η συνάρτηση πυκνότητας πιθανότητας της μεταβλητής θα
έχει ως εξής:

\begin{equation}
    P(x) = \frac{e^{-\lambda} \lambda^x}{x!}
\end{equation}

Το κοινό διάγραμμα παρουσίασης της κατανομής Poisson για διαφορετικές 
τιμές του $\lambda$ φαίνεται στο παρακάτω διάγραμμα:

\begin{center}
\includegraphics[scale=0.3]{../Images/task1ask1.png}
\end{center}

Ο κώδικας που χρησιμοποιήθηκε για την παραγωγή του διαγράμματος είναι παρακάτω και 
παραδόθηκε από τους διδάσκοντες:

\begin{lstlisting}[language=matlab]
# TASK_A: In a common diagram, design the Probability Mass Function (PMF)
# of the Poisson distribution with lambda parameters 7, 20, 40, 70. 
lambda = [7,20,40,70];

# For horizontal axis, choose k between 0 and 100, with step 1.
k = 0:1:100;

# Define the Poisson PMF values
for i=1:columns(lambda)
  poisson(i,:) = poisspdf(k,lambda(i));
endfor

# The colors for each plot-line are respectively:
colors = "rbkm";
figure(1);
hold on;

# Plotting the Poisson with parameter lambda, i.e. Po(lambda)
for i=1:columns(lambda)
  stem(k,poisson(i,:),colors(i),"linewidth",1.2);
endfor
hold off;
title("Probability Mass Function of the Poisson distribution");
xlabel("k values");
ylabel("Probability values");
legend("lambda = 7","lambda = 20","lambda = 40","lambda = 70");
\end{lstlisting}

Από το διάγραμμα είναι σαφές πως με την αύξηση της παραμέτρου $\lambda$
μειώνεται το ύψος της καμπύλης και μετατοπίζεται προς τα δεξιά. Γνωρίζουμε
από την θεωρία πως η μέση τιμή της κατανομής Poisson θα είναι ίση 
με την παράμετρο $\lambda$ και η διακύμανση θα είναι ίση με την παράμετρο
$\lambda$ επίσης. Η παράμετρος $\lambda$ αντιπροσωπεύει τον μέσο ρυθμό
εμφάνισης των γεγονότων στο χρόνο. Όσο αυξάνεται η τιμή της παραμέτρου
$\lambda$ τόσο μεγαλύτερη είναι η πιθανότητα εμφάνισης γεγονότων σε ένα
διάστημα χρόνου.

\subsection{Ερώτημα Β}
Μπορούμε να προσεγγίσουμε την κατανομή Poisson με την διωνυμική 
κατανομή ως το όριο της διωνυμικής κατανομής όταν ο αριθμός των
δοκιμών της διωνυμικής κατανομής τείνει προς το άπειρο και η πιθανότητα
επιτυχίας της διωνυμικής κατανομής τείνει προς το μηδέν όπου η παράμετρος 
$\lambda$ της κατανομής Poisson είναι ίση με το γινόμενο των δύο παραμέτρων
της διωνυμικής κατανομής. \\

Το διάγραμμα φαίνεται παρακάτω: 

\begin{center}
    \includegraphics[width=\textwidth]{../Images/task2ask1.png}
\end{center}

\pagebreak

Ο κώδικας που χρησιμοποιήθηκε για την παραγωγή του διαγράμματος είναι ο εξής:

\begin{lstlisting}[language=Matlab]
# TASK_B: Show that Poisson is the limit of the binomial distribution.
# Define the Poisson parameter lambda
lambda_constant = 40;

# Define the Binomial parameters n (number of trials) and p (probability for success)
# accordingly, for the approximation:
n = [40,120,360,1080,40000];
p = lambda_constant./n;

# Define the Binomial PMF values
for i=1:columns(n)
  binomial(i,:) = binopdf(k,n(i),p(i));
endfor

colors = "ygmbr";
figure(2);
subplot(2,1,1);
hold on;

# Plotting the Bino(n,p) for all the values of n
for i=1:columns(n)
  stem(k,binomial(i,:),colors(i),"linewidth",1.2);
endfor
title("PMF of the Binomial distribution");
xlabel("k values");
ylabel("Probability values");
legend();
hold off;
subplot(2,1,2);
hold on;

# In order to zoom in and notice the approximation
# better, we will work in the following way:

# Obtain the position of the Po(40) from the 1st Fig.
index = find(lambda == 40);
# Plot the Bino(n,p) without the first (n=40) value  
for i=2:columns(n)
  stem(k,binomial(i,:),colors(i),"linewidth",1.2);
endfor
# Include the Po(40) from the 1st Fig. with the same color
stem(k,poisson(index,:),"k","linewidth",1.2);
hold off;
title("Poisson as the limit of the Binomial distribution");
xlabel("k values");
ylabel("Probability values");
legend();
\end{lstlisting}

\subsection{Ερώτημα Γ}
Είναι γνωστό πως για μια τυχαία μεταβλητή η οποία ακολουθεί κατανομή Poisson η μέση τιμή 
και η διακύμανση της θα είναι ίση με την παράμετρο $\lambda$ της κατανομής. Θα το αποδείξουμε 
παρακάτω.

Η μέση τιμή έχει ως εξής:

\begin{equation}
    \begin{split}
        \mu = E[X] = & \sum_{x=0}^{\infty} x P(x) = \\
        & \sum_{x=0}^{\infty} x \frac{e^{-\lambda} \lambda^x}{x!} = \\
        & e^{-\lambda} \sum_{x=0}^{\infty} x \frac{\lambda^x}{x!} = \\
        & e^{-\lambda} \lambda e^{\lambda} = \\
        & \lambda
    \end{split}
\end{equation}

Γνωρίζουμε αρχίκα ότι ισχύει:

\begin{center}
    \begin{equation}
        \sigma^2 = E[X^2] - E[X]^2
    \end{equation}
\end{center}

και ότι η ροπή δεύτερης τάξης είναι ίση με:

\begin{center}
    \begin{equation}
        Ε[X^2] = E[X(X-1)] + E[X]
    \end{equation}
\end{center}

Αν υπολογίσουμε την ροπή δεύτερης τάξης της κατανομής Poisson θα έχουμε:\\

\begin{equation}
    \begin{split}
    E[X(X-1)] =
    & \sum_{x=0}^{\infty} k(k-1) P(k) = \\
    & \sum_{x=0}^{\infty} k(k-1) \frac{e^{-\lambda} \lambda^k}{k!} = \\
    & e^{-\lambda} \sum_{x=0}^{\infty} k(k-1) \frac{\lambda^k}{(k)!} = \\
    & \lambda^2 e^{-\lambda} \sum_{x=0}^{\infty} \frac{\lambda^{k-2}}{(k-2)!} = \\
    & \lambda^2 e^{-\lambda} e^{\lambda} = \lambda^2
    \end{split}
\end{equation}

Άρα προκύπτει:

\begin{equation}
        E[X^2] = & E[X(X-1)] + E[X] = \lambda^2 + \lambda
\end{equation}

Και επομένως:

\begin{equation}
        \sigma^2 = & E[X^2] - E[X]^2 = 
        \lambda^2 + \lambda - \lambda^2 = 
        \lambda
\end{equation}

Άρα η διακύμανση της κατανομής Poisson είναι ίση με την παράμετρο $\lambda$ της κατανομής
και για το παρόν παράδειγμα η διακύμανση της κατανομής Poisson είναι ίση με 40.

\pagebreak

\subsection{Ερώτημα Δ}


α) Είναι σαφές πως η κατανομή Poisson με $\lambda = 30$ είναι η υπέρθεση των δύο άλλων 
κατανομών Poisson με $\lambda_1 = 9$ και $\lambda_2 = 21$ όπου $\lambda = \lambda_1 + \lambda_2$.
Η Poisson αυτή μπορεί να διασπαστεί σε δύο διαφορετικές κατανομές Poisson με διαφορετικές
παραμέτρους $\lambda_1$ και $\lambda_2$ όπου $\lambda = \lambda_1 + \lambda_2$ και 
$p_{ext} = 0.3$ και $q_{int} = 0.7$ αντίστοιχα αφού $\lambda_1 = \lambda p_{ext}$ και
$\lambda_2 = \lambda q_{int}$. Το διάγραμμα φαίνεται παρακάτω:

\begin{center}
    \includegraphics[scale=0.3]{../Images/task4ask1.png}
\end{center}

β) Η πιθανότητα αυτή είναι ίση με $0.09$ και μπορεί να υπολογιστεί με την χρήση του
τύπου της διωνυμικής κατανομής όπως φαίνεται παρακάτω:

\begin{equation}
    P(X=k) = \binom{n}{k} p^k q^{n-k} = \binom{2}{2} 0.3^2 0.7^{2-2} = 0.09
\end{equation}


γ) Η πιθανότητα αυτή είναι ίση με $0.3087$ και μπορεί να υπολογιστεί με την χρήση του
τύπου της διωνυμικής κατανομής όπως φαίνεται παρακάτω:

\begin{equation}
    P(X=k) = \binom{n}{k} p^k q^{n-k} = \binom{5}{2} 0.3^2 0.7^{5-2} = 0.3087
\end{equation}


\pagebreak

Ο κώδικας που χρησιμοποιήθηκε φαίνεται παρακάτω

\begin{lstlisting}[language=Matlab]
# TASK_D: Poisson decomposition/superposition.

# a) Plot in a common diagram the 3 distributions

# Define the lambda parameter
lambdaX = 30;

# Define the Poisson PMF values regarding the number of calls
PMF_X = poisspdf(k,lambdaX);

# Define the probabilities regarding the calls (external, internal)
p_ext = 0.3;
q_int = 1-p_ext;

# Thus, each type of call-event will follow a Poisson distribution

# Define the lambda parameters:
lambdaX_1 = lambdaX*p_ext;
lambdaX_2 = lambdaX*q_int;

# Define the PMF values for each type of call
PMF_X1 = poisspdf(k,lambdaX_1);
PMF_X2 = poisspdf(k,lambdaX_2);

colors = "rbg";
figure(3);
hold on;

# Plotting the Poisson calls
stem(k,PMF_X1,colors(1),"linewidth",1.2);
stem(k,PMF_X2,colors(2),"linewidth",1.2);
stem(k,PMF_X,colors(3),"linewidth",1.2);
hold off;
title("Probability Mass Function of the Poisson calls");
xlabel("k values");
ylabel("Probability values");
# legend for each case
legend1 = sprintf('lambdaX_1 = %d', lambdaX_1);
legend2 = sprintf('lambdaX_2 = %d', lambdaX_2);
legend3 = sprintf('lambdaX = %d', lambdaX);
legend(legend1,legend2,legend3)

# Calculation of probabilities

# using the number of combinations a.k.a. "binomial coefficient"
# (or directly using the binopdf from the statistics pkg) 

# the probability (given n and k) can be found in either of the following ways:

# b)
# Define the number of n trials (number of calls made)
number_of_calls_b = 2;
# Define the k "successes", that, in our case, is
# the number of external calls with probability p_ext
k_ext_b = 2;
Prob_b = (nchoosek(number_of_calls_b,k_ext_b))*(p_ext^k_ext_b)*(q_int^(number_of_calls_b-k_ext_b));
# or directly with the use of binopdf (requires the statistics pkg)
Prob_b_bino = binopdf(k_ext_b,number_of_calls_b,p_ext);
display("The probability that both of the next two calls are external is");
display(Prob_b);

# c)
number_of_calls_c = 5;
k_ext_c = 2;
Prob_c = (nchoosek(number_of_calls_c,k_ext_c))*(p_ext^k_ext_c)*(q_int^(number_of_calls_c-k_ext_c));
# or, with the use of binopdf,
Prob_c_bino = binopdf(k_ext_c,number_of_calls_c,p_ext);
display("The probability that out of the 5 calls, two calls exactly are external, is");
display(Prob_c);
\end{lstlisting}

\pagebreak

\section{Εκθετική κατανομή}

\subsection{Ερώτημα Α}
Η εκθετική κατανομή ορίζεται ως εξής:

\begin{equation}
    P(x) = \frac{1}{\lambda} e^{-\frac{x}{\lambda}}
\end{equation}

Το διάγραμμα παρουσίασης της εκθετικής κατανομής για τις διαφορετικές τιμές του $\lambda$ φαίνεται
στο παρακάτω διάγραμμα:

\begin{center}
    \includegraphics[scale=0.3]{../Images/task1ask2.png}
\end{center}

Ο κώδικας που χρησιμοποιήθηκε για την παραγωγή του διαγράμματος είναι ο εξής:

\begin{lstlisting}[language=Matlab]
lambda = [0.5, 1.5, 4];

x = 0:0.001:8;

pdf(1,:) = exppdf(x,lambda(1));
pdf(2,:) = exppdf(x,lambda(2));
pdf(3,:) = exppdf(x,lambda(3));

figure(1);

hold on;

plot(x,pdf(1,:), 'r', 'LineWidth', 1.2);
plot(x,pdf(2,:), 'g', 'LineWidth', 1.2);
plot(x,pdf(3,:), 'b', 'LineWidth', 1.2);

hold off; 

xlabel('k values');
ylabel('Probability');
legend('lambda = 0.5', 'lambda = 1.5', 'lambda = 4');

\end{lstlisting}

\pagebreak

\subsection{Ερώτημα Β}
Γνωρίζουμε ότι $F(x) = P(X \leq x) = 1 - e^{-\frac{x}{\lambda}}$. Επίσης 
γνωρίζουμε ότι η κατανομή της τυχαίας μεταβλητής $X = min(X_1, X_2, X_3)$
είναι η εκθετική κατανομή με παράμετρο $\lambda = \lambda_1 + \lambda_2 + \lambda_3$.

H γραφική παράσταση των συναρτήσεων σε ίδιο διάγραμμα φαίνονται παρακάτω:

\begin{center}
    \includegraphics[scale=0.3]{../Images/task2ask2.png}
\end{center}

Ο κώδικας που χρησιμοποιήθηκε για την παραγωγή του διαγράμματος είναι ο εξής:

\begin{lstlisting}[language=Matlab]
lambdas = [0.5, 1.5, 4];

min_lambda = 0.5 + 1.5 + 4;

x = 0:00001:8;

for i=1:columns(lambdas)
	cdf(i,:) = expcdf(x,lambdas(i));
endfor

cdf_min = expcdf(x, min_lambda);

colors = 'rgbm';

figure(1);

hold on;

for i=1:columns(lambdas)
	plot(x, cdf(i,:), colors(i), 'LineWidth', 1.2);
endfor

plot(x, cdf_min, colors(4), 'LineWidth', 1.2);

hold off;

xlabel('k values');
ylabel('Probability');
legend('lambda1', 'lambda2', 'lambda3', 'min lambda');

\end{lstlisting}

Αυτό που παρατηρούμε σχετικά με τη μέση τιμή των εκθετικών κατανομών είναι ότι όσο
μεγαλύτερη είναι η τιμή της παραμέτρου $\lambda$ τόσο μικρότερη είναι η μέση τιμή της
κατανομής.



\subsection{Ερώτημα Γ}
Από τον ορισμό της δεσμευμένης πιθανότητας έχουμε: 

\begin{equation}
    \begin{split}
    P(X \geq x + t | X \geq t) =
    & \frac{P(X \geq x + t \cap X \geq t)}{P(X \geq t)} = \\
    & \frac{P(X \geq x + t)}{P(X \geq t)} = \\
    & \frac{1 - F(x + t)}{1 - F(t)} = \\
    & \frac{1 - (1 - e^{-\frac{x + t}{\lambda}})}{1 - (1 - e^{-\frac{t}{\lambda}})} = \\
    & \frac{e^{-\frac{x + t}{\lambda}}}{e^{-\frac{t}{\lambda}}} = e^{-\frac{x}{\lambda}} = 1 - F(x) = P(X \geq x)
\end{split}
\end{equation}

Επομένως εύκολα προκύπτει ότι:

\begin{equation}
    P(X > 45000 | X > 25000) = P(X > 20000 + 25000 | X > 25000) = P(X > 20000)
\end{equation}

Επομένως οι δύο πιθανότητες θα πρέπει να είναι ίσες όπως και επιβεβαιώνεται από τον κώδικα
που παρατίθεται παρακάτω:

\begin{lstlisting}[language=Matlab]
x = 0:0.001:8;

exp = expcdf(x,2.5);

p_1 = 1 - exp(20000);
display(p_1);

p_2 = (1 - exp(45000))./(1 - exp(25000));
display(p_2);
\end{lstlisting}

Οι πιθανότητες που υπολογίστηκαν είναι ίσες και ίσες με $0.92312$.



\pagebreak
\section{Άσκηση 3}

\subsection{Ερώτημα Α}
Η κατανομή που ακολουθείτε από τους χρόνους ανάμεσα στην εμφάνιση δύο διαδοχικών 
γεγονότων είναι μία εκθετική κατανομή με μέση τιμή $\frac{1}{\lambda}$.

Το διάγραμμα φαίνεται παρακάτω:

\begin{center}
    \includegraphics[scale=0.3]{../Images/task1ask3.png}
\end{center}

Το κώδικας που χρησιμοποιήθηκε για την παραγωγή του διαγράμματος είναι ο εξής:

\begin{lstlisting}[language=Matlab]
x = exprnd(0.1, 1,100);
y = ones(100,1);

 for i=1:99
  x(i+1) = x(i+1) + x(i);
  y(i+1) = y(i+1) + y(i);
endfor


figure(1); 

stairs(x, y, color='r');
xlabel('seconds');
ylabel('N(t)');
title('λ = 10');

\end{lstlisting}


\subsection{Ερώτημα Β}
Σε ένα τέτοιο χρονικό παράθυρο ακολουθείτε κατανομή Poisson με μέσο αριθμό γεγονότων
$\lambda \cdot \Delta \Tau$. Η παράμετρος $\lambda$ είναι προσεγγίσιμη από το πηλίκο 
του αριθμού των γεγονότων με το χρονικό παράθυρο $\Delta \Tau$.\\

Ο μέσος ρυθμός για το προηγούμενο ερώτημα είναι ίσος με $100/exprnd(100) = 11.813$.\\

\pagebreak

Για διαφορετικές τιμές αριθμού τυχαίων γεγονότων έχουμε: 
\begin{enumerate}
    \item{Για $Ν = 200$ έχουμε $\lambda = 9.6447$}
        \begin{center}
            \includegraphics[scale=0.4]{../Images/events_200.jpg}
        \end{center}
    \item{Για $Ν = 400$ έχουμε $\lambda = 9.6523$}
        \begin{center}
            \includegraphics[scale=0.4]{../Images/events_400.jpg}
        \end{center}
    \item{Για $Ν = 700$ έχουμε $\lambda = 9.9795$}
        \begin{center}
            \includegraphics[scale=0.4]{../Images/events_700.jpg}
        \end{center}
    \item{Για $Ν = 1000$ έχουμε $\lambda = 10.127$}
        \begin{center}
            \includegraphics[scale=0.4]{../Images/events_1000.jpg}
        \end{center}
    \item{Για $Ν = 5000$ έχουμε $\lambda = 10.133$}
        \begin{center}
            \includegraphics[scale=0.4]{../Images/events_5000.jpg}
        \end{center}
    \item{Για $Ν = 50000$ έχουμε $\lambda = 10.018$}
        \begin{center}
            \includegraphics[scale=0.4]{../Images/events_50000.jpg}
        \end{center}

    \pagebreak

    \item{Για $Ν = 100000$ έχουμε $\lambda = 9.9654$}
        \begin{center}
            \includegraphics[scale=0.4]{../Images/events_100000.jpg}
        \end{center}
\end{enumerate}

Ο κώδικας φαίνεται παρακάτω:

\begin{lstlisting}[language=Matlab]
ev = [200, 400, 700, 1000, 5000, 50000, 100000];

for j=1:columns(ev)
	x = exprnd(0.1, 1,ev(j));
	y = ones(ev(j),1);
	k = ev(j) - 1;
	for i=1:k
		x(i+1) = x(i+1) + x(i);
		y(i+1) = y(i+1) + y(i);
	endfor


	figure(1); 

	titlename = sprintf('amount of events %d', ev(j));

	stairs(x, y, color='r');
	xlabel('seconds');
	ylabel('N(t)');
	title(titlename);

	filename = sprintf('../images/events_%d.jpg', ev(j));
	print('-djpg',filename);

	solution = ev(j) / x(ev(j)); 
	display(ev(j));
	display(solution);

	
endfor

\end{lstlisting}

Από τα διαγράμματα φαίνεται ότι με την αύξηση των γεγονότων ο μέσος ρυθμός των γεγονότων
ταλαντώνεται γύρω από την τιμή του $\lambda = 10$ πλησιάζοντας διαρκώς όλο και πιο κοντά.
Η ευκρίνεια στα διαγράμματα επίσης αυξάνεται με την αύξηση των γεγονότων.










\end{document}
