\documentclass[12pt]{article}
\usepackage[greek,english]{babel}
\usepackage{alphabeta}
\usepackage{listings}
\usepackage{xcolor}
\usepackage{hyperref}
\usepackage{tabularx}
\usepackage{mathtools}
\usepackage{graphicx}
\usepackage{blindtext}
\usepackage{geometry}
\usepackage{listings}
\usepackage{amsmath}
\usepackage{amsfonts}
\usepackage{steinmetz}
\usepackage{algorithm}
\usepackage[noend]{algpseudocode}
\usepackage[shortlabels]{enumitem}
\usepackage{tikz}
\usepackage{fdsymbol}
\geometry{
    a4paper,
    total={170mm,257mm},
    left=20mm,
    top=20mm,
}

\author{Αυγερινός Πέτρος 03115074}
\title{Συστήματα Αναμονής Άσκηση 3^η}
\date{}

\begin{document}
\maketitle 
\pagebreak

\tableofcontents
\pagebreak

\section{Προσομοίωση 1}
\subsection{Ερώτημα 1}

Στο πρώτο ερώτημα ζητείται να υλοποιηθεί η προσομοίωση μίας 
ουράς M/M/1/10 για τρεις διαφορετικές τιμές του 
$\lambda = \{ 1, 5, 10\}$.

\begin{enumerate}
    \item{}
        Για το $\lambda = 1$ έχουμε τα παρακάτω αποτελέσματα:

        \begin{center}
            \includegraphics[scale=0.4]{../images/task1ask1_convergence_1.png}
        \end{center}

        \begin{center}
            \includegraphics[scale=0.4]{../images/task1ask1_probabilities_1.png}
        \end{center}

    \item{}
        Για το $\lambda = 5$ έχουμε τα παρακάτω αποτελέσματα:

        \begin{center}
            \includegraphics[scale=0.4]{../images/task1ask1_convergence_5.png}
        \end{center}

        \begin{center}
            \includegraphics[scale=0.4]{../images/task1ask1_probabilities_5.png}
        \end{center}

    \item{}
        Για το $\lambda = 10$ έχουμε τα παρακάτω αποτελέσματα:

        \begin{center}
            \includegraphics[scale=0.4]{../images/task1ask1_convergence_10.png}
        \end{center}

        \begin{center}
            \includegraphics[scale=0.4]{../images/task1ask1_probabilities_10.png}
        \end{center}
\end{enumerate}

\subsection{Ερώτημα 2}
Για την ταχύτητα σύγκλισης για κάθε διαφορετική τιμή του $\lambda$
έχουμε τα εξής:

\begin{enumerate}
    \item{}
        Για το $\lambda = 1$ έχουμε $Transitions = 37000$, $Delay = 0.25$ και $Blocking = 0$
    \item{}
        Για το $\lambda = 5$ έχουμε $Transitions = 113000$, $Delay = 1.127178$ και $Blocking = 0.095912$
    \item{}
        Για το $\lambda = 10$ έχουμε $Transitions = 400000$, $Delay = 1.810885$ και $Blocking = 0.502224$
\end{enumerate}

Βλέπουμε ότι με την αύξηση του $\lambda$ αυξάνεται και το πλήθος των
μεταβάσεων που χρειάζονται για να συγκλίνει το σύστημα. Αυτό σημαίνει
πως μειώνεται ο χρόνος σύγκλισης, δηλαδή απαιτείται περισσότερος χρόνος 
ώστε το σύστημα να βγει από τη μεταβατική κατάσταση και να έρθει σε 
ισορροπία. Αυτό συμβαίνει διότι με την αύξηση του $\lambda$ αυξάνεται
και το πλήθος των πελατών που εισέρχονται στο σύστημα, παράλληλα όμως
διατηρείται το $\mu$ σταθερό και προκύπτει bottleneck στο σύστημα το 
οποίο απαιτεί περισσότερο χρόνο για να απορροφήσει τους πελάτες.


\subsection{Ερώτημα 3}
Οι αλλαγές που πρέπει να γίνουν στον κώδικα είναι ότι κάθε φορά που 
αλλάζει το current\_state με κάποια προσθαφαίρεση, πρέπει να αλλάζει 
το $\mu$ σύμφωνα με τον τύπο $\mu_i = \mu (i + 1)$ και με αυτή την 
αλλαγεί να αλλάζει σαφώς και το threshold για αφίξεις και αναχωρήσεις.\\

Ο κώδικας για όλα τα ερωτήματα φαίνεται παρακάτω:

\lstinputlisting[language=matlab, basicstyle=\tiny, 
frame=single, backgroundcolor = \color{lightgray}]{../Code/task1ask1.m}

\section{Προσομοίωση 2}

Η μέση καθυστέρηση του συστήματος είναι 0.342785 ενώ η blocking πιθανότητα
είναι 0.004405.


Τα διαγράμματα για την δεύτερη προσομοίωση φαίνονται παρακάτω:

\begin{center}
    \includegraphics[scale=0.4]{../images/task1ask2_convergence_5.png}
\end{center}

\begin{center}
    \includegraphics[scale=0.4]{../images/task1ask2_probabilities_5.png}
\end{center}

Παρατηρώ ότι η ταχύτητα σύγκλισης είναι μικρότερη από ότι στο σύστημα 
της πρώτης προσομοίωσης για το ίδιο $\lambda$. Αυτό συμβαίνει διότι
στο σύστημα οι εξυπηρετητές "κουράζονται" και αυτό επηρεάζει την ταχύτητα
εξυπηρέτησης και την ταχύτητα σύγκλισης του συστήματος.

Ο κώδικας για τα παραπάνω φαίνεται παρακάτω:

\lstinputlisting[language=matlab, basicstyle=\tiny,
frame=single, backgroundcolor = \color{lightgray}]{../Code/task1ask2.m}

 






\end{document}
